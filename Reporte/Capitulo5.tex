\section{Motivación}
La creciente problemática de los accidentes de tráfico y la congestión vial en las grandes y pequeñas urbes impulsa la búsqueda de soluciones. La conducción autónoma busca mejorar la seguridad vial y optimizar la movilidad urbana. Reduciendo la intervención humana que es causante de diversos errores que terminan en accidentes automovilísticos.
Diversas entidades, investigadores y empresas han abordado el problema de la conducción autónoma desde diferentes perspectivas, es por ello que cada día es más común observar coches que cuentan con algún dispositivo de asistencia que permite escalar en los niveles de autonomía. A diferencia de los grandes recursos materiales y tecnológicos que cuentan estas entidades, es relevante aportar en el desarrollo de algoritmos de percepción, planeación y control para vehículos sin conductor donde el procesamiento ocurra en dispositivos embebidos, concretamente el uso de la placa ESP32, donde  es desafiante abordar un problema complejo donde la capacidad del hardware es limitada.


\section{Justificación}
Contribuir al desarrollo y la investigación en el campo de los coches autónomos con un enfoque específico en la integración de dispositivos embebidos. Al centrarnos sobre el uso de la ESP32, se plantea  una solución de \textit{edge computing} que permite el procesamiento de datos en el propio vehículo, agilizando los tiempos de respuesta y la toma de decisiones en situaciones criticas. Este enfoque permite evaluar el rendimiento y funcionamiento con dicho recurso, sin la necesidad de invertir en hardware costoso. No solo servirá como una investigación educativa, si no que al abordar problemas complejos de manera directa y práctica, se traduce en proponer un enfoque que pueda ser aplicado en entornos industriales, promoviendo un futuro donde los vehículos autónomos no sean un lujo sino una realidad accesible para mejorar la vida urbana.